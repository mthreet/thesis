%
%
% UCSD Doctoral Dissertation Template
% -----------------------------------
% http://ucsd-thesis.googlecode.com
%
%


%% REQUIRED FIELDS -- Replace with the values appropriate to you

% No symbols, formulas, superscripts, or Greek letters are allowed
% in your title.
\title{Separation of a Known Speaker's Voice with a Convolutional Neural Network}

\author{Michael Threet}
\degreeyear{\the\year}

% Master's Degree theses will NOT be formatted properly with this file.
\degreetitle{Master of Science}

\field{Electrical and Computer Engineering}
\specialization{Signal and Image Processing}  % If you have a specialization, add it here

\chair{Professor Truong Nguyen}
% Uncomment the next line iff you have a Co-Chair
% \cochair{Professor Cochair Semimaster}
%
% Or, uncomment the next line iff you have two equal Co-Chairs.
%\cochairs{Professor Chair Masterish}{Professor Chair Masterish}

%  The rest of the committee members  must be alphabetized by last name.
\othermembers{
Professor Humor Less\\
Professor Ironic Name\\
Professor Cirius Thinker\\
}
\numberofmembers{4} % |chair| + |cochair| + |othermembers|


%% START THE FRONTMATTER
%
\begin{frontmatter}

%% TITLE PAGES
%
%  This command generates the title, copyright, and signature pages.
%
\makefrontmatter

%% DEDICATION
%
%  You have three choices here:
%    1. Use the ``dedication'' environment.
%       Put in the text you want, and everything will be formated for
%       you. You'll get a perfectly respectable dedication page.
%
%
%    2. Use the ``mydedication'' environment.  If you don't like the
%       formatting of option 1, use this environment and format things
%       however you wish.
%
%    3. If you don't want a dedication, it's not required.
%
%
% \begin{dedication}
%   To two, the loneliest number since the number one.
% \end{dedication}


% \begin{mydedication} % You are responsible for formatting here.
%   \vspace{1in}
%   \begin{flushleft}
% 	To me.
%   \end{flushleft}
%
%   \vspace{2in}
%   \begin{center}
% 	And you.
%   \end{center}
%
%   \vspace{2in}
%   \begin{flushright}
% 	Which equals us.
%   \end{flushright}
% \end{mydedication}



%% EPIGRAPH
%
%  The same choices that applied to the dedication apply here.
%
%\begin{epigraph} % The style file will position the text for you.
%  \emph{A careful quotation\\
%  conveys brilliance.}\\
%  ---Smarty Pants
%\end{epigraph}

% \begin{myepigraph} % You position the text yourself.
%   \vfil
%   \begin{center}
%     {\bf Think! It ain't illegal yet.}
%
% 	\emph{---George Clinton}
%   \end{center}
% \end{myepigraph}


%% SETUP THE TABLE OF CONTENTS
%
\tableofcontents
\listoffigures  % Comment if you don't have any figures
\listoftables   % Comment if you don't have any tables



%% ACKNOWLEDGEMENTS
%
%  While technically optional, you probably have someone to thank.
%  Also, a paragraph acknowledging all coauthors and publishers (if
%  you have any) is required in the acknowledgements page and as the
%  last paragraph of text at the end of each respective chapter. See
%  the OGS Formatting Manual for more information.
%
\begin{acknowledgements}
 Thanks to whoever deserves credit for Blacks Beach, Porters Pub, and
 every coffee shop in San Diego.

 Thanks also to hottubs.
\end{acknowledgements}


%% VITA
%
%  A brief vita is required in a doctoral thesis. See the OGS
%  Formatting Manual for more information.
%
%\begin{vitapage}
%\begin{vita}
%  \item[2002] B.~S. in Mathematics \emph{cum laude}, University of Southern North Dakota, Hoople
%  \item[2002-2007] Graduate Teaching Assistant, University of California, San Diego
%  \item[2007] Ph.~D. in Mathematics, University of California, San Diego
%\end{vita}
%\begin{publications}
%  \item Your Name, ``A Simple Proof Of The Riemann Hypothesis'', \emph{Annals of Math}, 314, 2007.
%  \item Your Name, Euclid, ``There Are Lots Of Prime Numbers'', \emph{Journal of Primes}, 1, 300 B.C.
%\end{publications}
%\end{vitapage}


%% ABSTRACT
%
%  Doctoral dissertation abstracts should not exceed 350 words.
%   The abstract may continue to a second page if necessary.
%
\begin{abstract}
	Source Separation (SS) refers to a problem in signal processing where two or more mixed signal sources must be separated into their individual components. While SS is a challenging problem, it may be simplified by making assumptions on the signals that are present and the methods used to mix the signals. One example of this is to limit the range of signals to human voices and to limit the total number of speakers (through either estimation or always having a set number of speakers). This paper assumes that the speech from two speakers is mixed at one microphone, with the voice of one speaker (Speaker 1) being present in all recordings. Traditional approaches to the SS problem typically involve array processing and time-frequency methods to perform the separation. Once such example is Non-Negative Matrix Factorization (NMF), which attempts to factor a spectrogram into frequency basis vectors and time weights for each speaker. This paper will explore the use of a neural network (NN) to learn effective separation of Speaker 1's voice from a variety of other speakers and background noises. The NN will prove to be much more effective than NMF due to the ability of the NN to learn a representative feature space of Speaker 1's speech.
\end{abstract}


\end{frontmatter}
